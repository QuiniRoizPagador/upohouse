\newpage{\pagestyle{empty}\cleardoublepage}
\newpage
\vspace*{\fill}
    \begin{center}
      \thispagestyle{empty} \vspace*{0cm} \textbf{\huge
Dise\~{n}o}
    \end{center}
    \vspace*{\fill}
\newpage{\pagestyle{empty}\cleardoublepage}
\chapter{Tabla Crud}


\begin{figure}[h]
\centering
\includegraphics[width=.7\textwidth]{Img/Disenyo/TABLA_CRUD.jpg}
\caption{Tabla CRUD con las entidades y sus operaciones cubiertas.}
\label{fig:dcu}
\end{figure}

\chapter{Puntos donde se manipulan las entidades}

\section{Usuario}
\subsection{Crear}

\begin{itemize}
\item Caso 1: Registro
\begin{itemize}
\item Localizaci\'{o}n
\begin{itemize}
\item Vista: view/loginView.php
\item Controlador: controller/UserController.php
\item Modelo: model/UserModel.php
\item Dao: model/dao/UserDao.php
\item Entidad: model/dao/dto/User.php
\end{itemize}
\item Acceso: Desde la pantalla principal hacer clic en iniciar sesi\'{o}n. En la siguiente pantalla hacer clic en ''Registro''. Una vez en el formulario de alta de usuario rellenar los campos y hacer clic en el bot\'{o}n registro.
\end{itemize}

\item Caso 2: Creaci\'{o}n desde dashboard
\begin{itemize}
\item Localizaci\'{o}n
\begin{itemize}
\item Vista: view/users.php
\item Controlador: controller/AdminController.php
\item Modelo: model/UserModel.php
\item Dao: model/dao/UserDao.php
\item Entidad: model/dao/dto/User.php
\end{itemize}
\item Acceso: Desde la vista de administrador, hacer clic en la llave inglesa que nos llevar\'{a} al dashboard del sistema en la opci\'{o}n Usuarios. Hacer clic en el ''+''. Una vez se despliegue el modal , rellenar los campos y hacer clic en el bot\'{o}n enviar.
\end{itemize}
\end{itemize}

\subsection{Modificar}
\begin{itemize}
\item Caso 1: Bloqueo desde dashboard Usuarios
\begin{itemize}
\item Localizaci\'{o}n
\begin{itemize}
\item Vista: view/users.php
\item Controlador: controller/AdminController.php
\item Modelo: model/UserModel.php
\item Dao: model/dao/UserDao.php
\item Entidad: model/dao/dto/User.php
\end{itemize}
\item Acceso: Desde la vista de administrador, hacer clic en la llave inglesa que nos llevar\'{a} al dashboard del sistema en la opci\'{o}n Usuarios. Hacer clic al bot\'{o}n de la columna Ver del registro que queramos modificar para consultarlo previamente, hacemos clic en el bot\'{o}n bloquear y confirmamos en el modal haciendo clic en bloquear.
\end{itemize}
\item Caso 2: Modificaci\'{o}n desde dashboard
\begin{itemize}
\item Localizaci\'{o}n
\begin{itemize}
\item Vista: view/users.php
\item Controlador: controller/AdminController.php
\item Modelo: model/UserModel.php
\item Dao: model/dao/UserDao.php
\item Entidad: model/dao/dto/User.php
\end{itemize}
\item Acceso: Desde la vista de administrador, hacer clic en la llave inglesa que nos llevar\'{a} al dashboard del sistema en la opci\'{o}n Usuarios. Hacer clic al bot\'{o}n de la columna Ver del registro que queramos modificar para consultarlo previamente, hacemos clic en el bot\'{o}n editar y modificamos los campos del formulario. Posteriormente haremos clic en el bot\'{o}n enviar.
\end{itemize}
\item Caso 3: Aceptar denuncia sobre usuario
\begin{itemize}
\item Localizaci\'{o}n
\begin{itemize}
\item Vista: view/denuncias.php
\item Controlador: controller/AdminController.php
\item Modelo: model/UserModel.php
\item Dao: model/dao/UserDao.php
\item Entidad: model/dao/dto/User.php
\end{itemize}
\item Acceso: Desde la vista de administrador, hacer clic en la llave inglesa que nos llevar\'{a} al dashboard del sistema en la opci\'{o}n Denuncias. Hacer clic al bot\'{o}n de la columna Ver del registro para consultar la denuncia, hacemos clic en el bot\'{o}n aceptar para modificar el estado del usuario a bloqueado.
\end{itemize}
\item Caso 4: Desbloqueo desde dashboard Usuarios
\begin{itemize}
\item Localizaci\'{o}n
\begin{itemize}
\item Vista: view/users.php
\item Controlador: controller/AdminController.php
\item Modelo: model/UserModel.php
\item Dao: model/dao/UserDao.php
\item Entidad: model/dao/dto/User.php
\end{itemize}
\item Acceso: Desde la vista de administrador, hacer clic en la llave inglesa que nos llevar\'{a} al dashboard del sistema en la opci\'{o}n Usuarios. Hacer clic al bot\'{o}n de la columna Ver del registro que queramos modificar para consultarlo previamente, hacemos clic en el bot\'{o}n desbloquear y confirmamos en el modal haciendo clic en desbloquear.
\end{itemize}
\end{itemize}

\subsection{Eliminar} 
\begin{itemize}
\item Localizaci\'{o}n
\begin{itemize}
\item Vista: view/users.php
\item Controlador: controller/AdminController.php
\item Modelo: model/UserModel.php
\item Dao: model/dao/UserDao.php
\item Entidad: model/dao/dto/User.php
\end{itemize}
\item Acceso: Desde la vista de administrador, hacer clic en la llave inglesa que nos llevar\'{a} al dashboard del sistema en la opci\'{o}n Usuarios. Hacer clic al bot\'{o}n de la columna Ver del registro que queramos eliminar para consultarlo previamente, hacemos clic en el bot\'{o}n eliminar y confirmamos en el modal haciendo clic en eliminar.
\end{itemize}

\subsection{Consultar}
\begin{itemize}
\item Caso 1: Consultar desde dashboard Usuarios
\begin{itemize}
\item Localizaci\'{o}n
\begin{itemize}
\item Vista: view/users.php
\item Controlador: controller/AdminController.php
\item Modelo: model/UserModel.php
\item Dao: model/dao/UserDao.php
\item Entidad: model/dao/dto/User.php
\end{itemize}
\item Acceso: Desde la vista de administrador, hacer clic en la llave inglesa que nos llevar\'{a} al dashboard del sistema en la opci\'{o}n Usuarios. Hacer clic al bot\'{o}n de la columna Ver del registro que queramos consultar.
\end{itemize}
\item Caso 2: Consultar desde link de usuario
\begin{itemize}
\item Localizaci\'{o}n
\begin{itemize}
\item Url: index/user/readUser\&uuid=XXXXXXXXXXX
\item Vista: view/profileView.php
\item Controlador: controller/UserController.php
\item Modelo: model/UserModel.php
\item Dao: model/dao/UserDao.php
\item Entidad: model/dao/dto/User.php
\end{itemize}
\item Acceso: Desde cualquier parte de la web desde donde se haga referencia a un usuario, se har\'{a} click en \'{e}l y este consultar\'{a} el mismo.
\end{itemize}
\end{itemize}

\section{Anuncio}
\subsection{Crear}
\begin{itemize}
\item Localizaci\'{o}n
\begin{itemize}
\item Vista: view/listAdsView.php
\item Controlador: controller/AdController.php
\item Modelo: model/AdModel.php
\item Dao: model/dao/AdDao.php
\item Entidad: model/dao/dto/Ad.php
\end{itemize}
\item Acceso: Desde cualquier p\'{a}gina hacemos click al apartado de anuncios de la cabecera del sistemas, hacemos click en el ''+'' y rellenamos los datos del formulario. Posteriormente haremos click en enviar.
\end{itemize}
\subsection{Modificar}
\begin{itemize}
\item Caso 1: Modificar desde consulta de anuncio
\begin{itemize}
\item Localizaci\'{o}n
\begin{itemize}
\item Vista: view/modifyAd.php
\item Controlador: controller/AdController.php
\item Modelo: model/AdModel.php
\item Dao: model/dao/AdDao.php
\item Entidad: model/dao/dto/Ad.php
\end{itemize}
\item Acceso: Desde el dashboard o desde la b\'{u}squeda de anuncios hacemos clic en el bot\'{o}n modificar, rellenamos los campos del formulario y hacemos click en enviar.
\end{itemize}
\item Caso 2: Bloquear anuncio
\begin{itemize}
\item Localizaci\'{o}n
\begin{itemize}
\item Vista: view/readAd.php
\item Controlador: controller/AdController.php
\item Modelo: model/AdModel.php
\item Dao: model/dao/AdDao.php
\item Entidad: model/dao/dto/Ad.php
\end{itemize}
\item Acceso: Desde el dashboard o desde la b\'{u}squeda de anuncios consultamos el anuncio deseado, hacemos clic en el bot\'{o}n bloquear y aceptamos en el modal para confirmar el bloqueo.
\end{itemize}
\item Caso 3: Desbloquear anuncio
\begin{itemize}
\item Localizaci\'{o}n
\begin{itemize}
\item Url: index/Ad/read\&uuid=XXXXXXXX
\item Vista: view/readAd.php
\item Controlador: controller/AdController.php
\item Modelo: model/AdModel.php
\item Dao: model/dao/AdDao.php
\item Entidad: model/dao/dto/Ad.php
\end{itemize}
\item Acceso: Desde el dashboard consultamos el anuncio, hacemos clic en el bot\'{o}n desbloquear.
\end{itemize}
\item Caso 4: Aceptar denuncia anuncio
\begin{itemize}
\item Localizaci\'{o}n
\begin{itemize}
\item Vista: view/denuncias.php
\item Controlador: controller/AdminController.php
\item Modelo: model/AdModel.php
\item Dao: model/dao/AdDao.php
\item Entidad: model/dao/dto/Ad.php
\end{itemize}
\item Acceso: Desde la vista de administrador, hacer clic en la llave inglesa que nos llevar\'{a} al dashboard del sistema en la opci\'{o}n Denuncias. Hacer clic al bot\'{o}n de la columna Ver del registro para consultar la denuncia, hacemos clic en el bot\'{o}n aceptar para modificar el estado del anuncio a bloqueado.
\end{itemize}
\end{itemize}

\subsection{Eliminar}
\begin{itemize}
\item Localizaci\'{o}n
\begin{itemize}
\item Url: index/Ad/read\&uuid=XXXXXXXX
\item Vista: view/readAd.php
\item Controlador: controller/AdController.php
\item Modelo: model/AdModel.php
\item Dao: model/dao/AdDao.php
\item Entidad: model/dao/dto/Ad.php
\end{itemize}
\item Acceso: Desde la vista de administrador de siendo propietario del anuncio lo consultaremos previamente, haremos click en el bot\'{o}n eliminar y confirmaremos la eliminaci\'{o}n.
\end{itemize}
\subsection{Consultar}
\begin{itemize}
\item Localizaci\'{o}n
\begin{itemize}
\item Url: index/Ad/read\&uuid=XXXXXXXX
\item Vista: view/readAd.php
\item Controlador: controller/AdController.php
\item Modelo: model/AdModel.php
\item Dao: model/dao/AdDao.php
\item Entidad: model/dao/dto/Ad.php
\end{itemize}
\item Acceso: Desde la vista de administrador o b\'{u}squeda de anuncios haremos click en el anuncio que queramos consultar.
\end{itemize}