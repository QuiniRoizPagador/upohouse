\chapter{Usuario Registrado}

\section{P\'{a}gina Principal}
El usuario que haya iniciado sesi\'{o}n ser\'{a} redireccionado autom\'{a}ticamente a la p\'{a}gina principal de la web, en la cual podr\'{a} observar que pr\'{a}cticamente no ha cambiado, excepto por su barra de navegaci\'{o}n, que le acompa\~{n}ar\'{a} en todo el sitio web. \\

Aparece una nueva opci\'{o}n de navegaci\'{o}n, la cual permite visitar el perfil de usuario y consultar sus peticiones sobre anuncios publicados. Tambi\'{e}n se ha sustitu\'{i}do el inicio de sesi\'{o}n por el cierre de sesi\'{o}n, para poder salir del sistema.

\begin{figure}[h!]
\centering
\includegraphics[width=1\textwidth]{Img/ManualUsuario/USER_MENU.png}
\end{figure}

Puede observarse que, en la parte izquierda, el usuario recibe un mensaje de bienvenida personalidado.

\begin{figure}[h!]
\centering
\includegraphics[width=.5\textwidth]{Img/ManualUsuario/USER_MENU_SALUDO.png}
\end{figure}

En lo referente a las b\'{u}squedas y listados de anuncios, las vistas no cambian con respecto al usuario no registrado, de manera que se proceder\'{a} a comentar las diferencias con respecto al anterior.

\section{Anuncios}
En las consultas propias de los anuncios se pueden apreciar una serie de caracter\'{i}sticas diferentes a las ofrecidas para un usuario no logueado en el sistema, o GUEST.\\

Aparte de que los datos del anuncio se muestren de la misma manera, las opciones del men\'{u} lateral ofrecen ahora la posibilidad de denunciar el anuncico, contactar con el anunciante realizando una petici\'{o}n, as\'{i} como valorar positiva o negativamente el  mismo. En este caso, el usuario registrado podr\'{a} consultar el perfil de usuario del anunciante haciendo click en su nombre.

\begin{figure}[h!]
\centering
\includegraphics[width=.3\textwidth]{Img/ManualUsuario/LATERAL_AD_USER_NO_PETICION.png}
\end{figure}

Una vez se haya realizado una petici\'{o}n sobre un anuncio, esta opci\'{o}n se ver\'{a} sustitu\'{i}da por el siguiente mensaje:
\begin{figure}[h!]
\centering
\includegraphics[width=.3\textwidth]{Img/ManualUsuario/LATERAL_AD_USER_PETICION.png}
\end{figure}

En la secci\'{o}n de mensajes, ahora se permite que se env\'{i}e un comentario sobre el anuncio. Este comentario ser\'{a} p\'{u}blico, por tanto, se podr\'{a} denunciar por otros usuarios del mismo modo que el usuario en cuesti\'{o}n pueda realizar una petici\'{o}n de denuncia sobre aquel comentario que desee.


\begin{figure}[h!]
\centering
\includegraphics[width=1\textwidth]{Img/ManualUsuario/USER_AD_COMMENTS.png}
\end{figure}

\section{Perfil de Usuario}

Al acceder desde la opci\'{o}n del men\'{u} principal con el nombre 'Perfil' el usuario dispondr\'{a} de un peque\~{n}o formulario con sus datos editables y una peque\~{n}a columna que muestre las actividades, como la cantidad de anuncios publicados y comentarios. A diferencia de los comentarios, los anuncios podr\'{a}n  ser consultados, list\'{a}ndose del mismo modo que en la p\'{a}gina de listado, en caso de existir m\'{a}s de uno, haciendo click sobre la opci\'{o}n pertinente. 


\begin{figure}[h!]
\centering
\includegraphics[width=1\textwidth]{Img/ManualUsuario/USER_PROFILE_EDIT.png}
\end{figure}

Justo encima de la tarjeta de perfil, se encuentra la de solicitudes, donde el usuario podr\'{a} listar y gestionar sus solicitudes existentes sobre alg\'{u}n anuncio que hubiera publicado.

\begin{figure}[h!]
\centering
\includegraphics[width=.7\textwidth]{Img/ManualUsuario/USER_REQUESTS.png}
\end{figure}

Haciendo click en el nombre del anuncio se redirigir\'{a} al anuncio en cuesti\'{o}n para poder observarlo. Del mismo modo, al hacer click en un usuario, se podr\'{a} ver el perfil del mismo.


\begin{figure}[h!]
\centering
\includegraphics[width=.8\textwidth]{Img/ManualUsuario/PERFIL_OTHER.png}
\end{figure}

Encima del nombre, a la derecha, aparecer\'{a} un bot\'{o}n que permite denunciar al usuario en caso de mal comportamiento. Se mostrar\'{a} para ello un formulario com\'{u}n a todas las tipos de denuncias, con t\'{i}tulo y descripci\'{o}n de la misma.

\begin{figure}[h!]
\centering
\includegraphics[width=.4\textwidth]{Img/ManualUsuario/DENUNCIA.png}
\end{figure}



\begin{figure}[h!]
\centering
\includegraphics[width=.4\textwidth]{Img/ManualUsuario/USER_REQUEST_READ.png}
\end{figure}
