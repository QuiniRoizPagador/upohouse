\newpage{\pagestyle{empty}\cleardoublepage}
\newpage
\vspace*{\fill}
    \begin{center}
      \thispagestyle{empty} \vspace*{0cm} \textbf{\huge
Manual de Usuario}
    \end{center}
    \vspace*{\fill}
\newpage{\pagestyle{empty}\cleardoublepage}
\chapter{Usuario no logueado}

\section{P\'{a}gina Principal}
De ahora en adelante GUEST, el usuario no logueado dispondr\'{a} de la misma vista principal que los usuarios del sistema, salvo el men\'{u} de navegaci\'{o}n, el cual mostrar\'{a} opciones extra a los usuarios que inicien sesi\'{o}n.\\


% meter aquí imagen de principal

En primer lugar se puede observar una barra de navegaci\'{o}n, en la cual se aprecia una imagen de una casa. Si se hace click en ella, se llevar\'{a} a esta p\'{a}gina principal. En la zona central del men\'{u} se encuentra el acceso al listado de todos los anuncios del sistema, para poder listarlos y filtrarlos correctamente. Tras esto, la opci\'{o} de iniciar sesi\'{o} y un desplegable que ofrecer\'{a} un cambio de idioma, entre espa\~{n}ol e ingl\'{e}s.

% meter imagen de barra de navegación


En la parte superior de la p\'{a}gina principal, el usuario GUEST podr\'{a} realizar una b\'{u}squeda global escribiendo las palabras claves por las que quiere buscar, y se le listar\'{a} un m\'{a}ximo de 10 resultados en tiempo real. Al seleccionar uno de los anuncios se acceder\'{a} a su respectiva vista, donde se podr\'{a} consultar todo dato relacionado con el mismo, as\'{i} como ver sus fotos detenidamente.

% aquí meter imagen de desplegable global search


En caso de que el usuario GUEST decida pulsar intro al escribir las palabras claves en el buscador, se le redirigir\'{a} a la siguiente p\'{a}gina con una b\'{u}squeda global paginada, en la cual se podr\'{a}n ver tarjetitas con los anuncios resultantes de la b\'{u}squeda.

% meter imagen de listado de tarjetas del buscador

Siguiendo en la p\'{a}gina principal, debajo del buscador existente se encuentran listadas 9 tarjetas con los \'{u}ltimos anuncios creados en el sistema, donde el usuario por\'{a} seleccionar alguna y acceder a la p\'{a}gina del anuncio en cuesti\'{o}n.

% meter imagen de las tarjetitas


Debajo de estas tarjetas, y com\'{u}n a todas las p\'{a}ginas se puede encontrar un footer con la informaci\'{o}n de los desarrolladores, as\'{i} como algunos links interesantes para facilitar la navegaci\'{o}n del usuario desde abajo.

% meter imagen del footer

\section{Registro e Inicio de Sesi\'{o}n}
En la pantalla de inicio de sesi\'{o}n se puede encontrar un formulario sencillo que solicita las credenciales para entrar en el sistema. Con usuario o email y la contrase\~{n}a adecuados, el usuario podr\'{a} identificarse en el sistema y acceder a las dem\'{a}s opciones del mismo.


% imagen del login


Para poder iniciar sesi\'{o}n, el usuario GUEST deber\'{a} registrarse, haciendo click en la pesta\~{n}a con el t\'{i}tulo 'Registro'. En \'{e}sta, se le ofrecer\'{a} un formulario con 7 campos, los cuales solicitan nombre, apellidos, email, tel\'{e}fono, login y contrase\~{n}a (dos veces para verificar).

% imagen del registro

En caso de existir alg\'{u}n fallo en alguno de estos formularios, se le mostrar\'{a} al usuario un mensaje similar al siguiente:

% imagen de error


\section{Anuncios}
Habiendo accedido a la secci\'{o}n del men\'{u} con el t\'{i}tulo 'Anuncios', se presenta una p\'{a}gina con un men\'{u} a la izquierda que permite filtrar los resultados mostrados en funci\'{o}n del tipo de operaci\'{o}n y el tipo de casa. Del mismo modo, los resultados se mostrar\'{a}n paginados, de forma que se podr\'{a} navegar entre los muchos anuncios que puedan coincidir con el criterio de filtro aplicado.

%-- Imagen de página de listado de ads


Al hacer click en un ad se apreciar\'{a} el siguiente contenido:


% imagen de consulta de anuncio
