\newpage{\pagestyle{empty}\cleardoublepage}
\newpage
\vspace*{\fill}
    \begin{center}
      \thispagestyle{empty} \vspace*{0cm} \textbf{\huge
Casos de Prueba y Resumen Ejecutivo}
    \end{center}
    \vspace*{\fill}
\newpage{\pagestyle{empty}\cleardoublepage}

\chapter{Casos de Prueba}

En los siguientes puntos se van a detallar aquellos casos de prueba relacionados con la seguridad que se han llevado a cabo, as\'{i} como aquellos que no se han podido solventar o est\'{a}n pendientes de arreglarse.

\section{Soluciones a vulnerabilidades}
\begin{itemize}


\item Siguiendo la arquitectura del sistema en cuesti\'{o}n, las consultas a la base de datos han aplicado en todas sus sentencias con datos externos y susceptibles de inyecci\'{o}n sql las sentencias de tipo Prepared Statement, mecanismo que ayudar\'{a} a evitar este tipo de ataques. (model/dao/NombreEntidadDao.php, extendiendo de core/AbstractDao.php).

A su vez, en la capa del controlador, y haciendo uso de core/RegularUtils.php, se sanear\'{a}n las cadenas de caracteres que reciban.

\item Las inyecciones XSS han sido prevenidas tambi\'{e}n saneando en el controlador, y utilizando nuevamente la clase RegularUtils.php creada para esa funcionalidad, adem\'{a}s de validaciones, expresiones regulares y generaci\'{o}n de c\'{o}digos UUID.

\item Los ficheros subidos son controlados en su tama\~{n}o y formato a partir de una variable en config/globals.php, la cual almacenar\'{a} el tama\~{n}o m\'{a}ximo permitido de subida, as\'{i} como su formato. 

\item El control de sesi\'{o}n se realiza en index.php, archivo php que llamar\'{a} siempre a las comprobaciones previas a redirecciones y cargas de controladores.

\item Tambi\'{e}n se ha controlado el acceso restringido seg\'{u}n el rol del tipo de usuario, siendo el usuario invitado un usuario no contemplado en la base de datos y con sus propias vistas restringidas (login y registro). Esta verificaci\'{o}n se contempla en el archivo Security.func.php, el cual es llamado desde FrontController.func.php, a su vez llamado desde index.php a la hora de verificar sesiones y redireccionamientos. Los accesos a las diferentes funciones del sistema se controlar\'{a}n a partir de la variable global ACTIONS declarada en config/global.php. 

\item Se ha decidido redireccionar a la p\'{a}gina principal para no dar informaci\'{o}n de posibles p\'{a}ginas privadas, prohibidas o no encontradas o hasta errores internos del servidor desde una funci\'{o}n 'lanzarAccion' en FrontController.func.php y en su propia llamada de index.php.

\item Las contrase\~{n}as vac\'{i}as y datos o campos vac\'{i}os no ser\'{a}n permitidos en el sistema, siendo controlados en el controlador pertinente usando el m\'{e}todo de validaci\'{o}n 'filtrarPorTipo' de RegularUtils.php.

\item Se han abstra\'{i}do al usuario de la url a trav\'{e}s de url's amigables, las cuales no dejar\'{a}n ver f\'{a}cilmente el tipo de sistema que lleva el servidor (en este caso php). Esto ha sido implementado gracias al archivo .htaccess en el lado del servidor. Adem\'{a}s, tambi\'{e}n ha sido utilizado para restringir acceso a carpetas no deseadas.

\item Si, por malas intenciones, alg\'{u}n usuario deshabilita las funciones de javascript del lado del cliente a la hora de enviar un formulario, el servidor validará y regresará a la página anterior con el error a mostrar o redirigirá a la página principal.

\end{itemize}
\section{Vulnerabilidades no controladas}

\begin{itemize}
\item No se han protegido los scripts AJAX del acceso, debido a que no tratan con informaci\'{o}n privilegiada o sensible.
\item La seguridad propia del servidor y la base de datos se han delegado al servicio contratado, por tanto no se han tenido en cuenta en esta documentaci\'{o}n.
\end{itemize}



